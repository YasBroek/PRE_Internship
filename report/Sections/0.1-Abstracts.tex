\selectlanguage{english}

\begin{abstract}
Multi-Agent Pathfinding (MAPF) is the problem of computing collision-free paths for multiple agents moving in a shared environment, usually represented as a graph, while seeking efficient overall movement. It has been extensively studied in the last years, given its theoretical interest and real-world applications. Few approaches of using Machine Learning for improving path computations in MAPF problems have been developed, given the difficulty related to the lack of differentiability in pathfinding methods. This report proposes ML models that apply Fenchel-Young Losses for allowing Gradient Descent application while learning from pathfinding algorithms.
\end{abstract} \hspace{10pt}
\keywords{Multi-Agent Pathfinding, Optimization, Machine Learning, Decision-Focused Learning, Fenchel-Young Losses}

\selectlanguage{french}

\begin{abstract}
Le Multi-Agent Pathfinding (MAPF) est le problème consistant à calculer des trajectoires sans collision pour plusieurs agents évoluant dans un environnement partagé, généralement représenté sous forme de graphe, tout en recherchant une efficacité globale des déplacements. Il a été largement étudié ces dernières années, en raison de son intérêt théorique et de ses applications réelles. Peu d'approches utilisant l'Apprentissage Automatique pour améliorer le calcul de trajectoires dans les problèmes de MAPF ont été développées, en raison de la difficulté liée au manque de différentiabilité des méthodes de recherche de chemins. Ce rapport propose des modèles d'Apprentissage Automatique qui appliquent les pertes de Fenchel-Young afin de permettre l'utilisation de la Descente de Gradient lors de l'apprentissage à partir d'algorithmes de recherche de chemins.
\end{abstract}\hspace{10pt}

\renewcommand{\keywords}[1]
{
  \small	
  \textbf{\textit{Mots clés---}} #1
}

\keywords{Multi-Agent Pathfinding, Optimisation, Apprentissage automatique, Apprentissage centré sur la décision, Fenchel-Young Losses}

\selectlanguage{english}