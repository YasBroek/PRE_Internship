\section{Conclusion}

During this internship, we were able to develop a Machine Learning model capable of reducing costs on paths computed using Prioritized Planning for some instances of the MAPF problem.

Although the developed models showed to be promising, there are some possible improvements that could improve even more their impact over the objective function. For one, further exploring the differences of impacts of applying ISP and PP in the training pipeline, as well as ellaborationg a model capable of directly learning by applying PP. The use of heuristics for building a hierarchic order to be applied in Prioritized Planning when computing final paths has the potential of improving greatly the results, and would be considered as the next step on a continuation of this presented study.

Other aspects that may be explored in future papers were also addressed, such as size and diversity of training sets and representability of feature extraction.

This internship was for me the opportunity of having a first contact with the research world and learning to use a number of different tools that will be very useful in the future, as well as to further explore utensils I was already familiar with. I was also able to develop my critic sense, comprehending how to search for and read articles, overcoming problems both in programming and in theory, and critically analysing results, searching to really understand the outputs of experiments, by breaking a problem into smaller pieces.

I was also able to deepen my knowledge in coding, using specifically the Julia language, and in Optimization, by becoming familiar with a number of different methods presented in articles, mostly related to alternatives of using Machine Learning for optimizing a Linear Program.